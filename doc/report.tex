\title{Relazione di ``Metodi del Calcolo Scientifico''}
\author{
	Simon Vocella \\
	Matricola: 718289
}
\date{\today}

% TODO [LD] scrivere una relazione che riporti un minimo di teoria
% TODO [LD] i listati dei programmi
% TODO [LD] gli esperimenti
% TODO [LD] le vostre considerazioni

% TODO [DCT2] scrivere una relazione che riporti un minimo di teoria
% TODO [DCT2] i listati dei programmi
% TODO [DCT2] gli esperimenti
% TODO [DCT2] le vostre considerazioni

\documentclass[12pt]{article}
\usepackage[margin=1.5cm]{geometry}
\usepackage[italian]{babel}
\usepackage{booktabs}
\usepackage{algorithm}% http://ctan.org/pkg/algorithm
\usepackage{algpseudocode}% http://ctan.org/pkg/algorithmicx
\usepackage{algorithmicx}
\usepackage{lipsum}% http://ctan.org/pkg/lipsum
\usepackage{float}% http://ctan.org/pkg/float
\usepackage{graphicx}
\usepackage{amsfonts}
\usepackage{amsmath}

\usepackage{listings}
\lstset{language=Java, basicstyle=\small}
\lstset{linewidth=\textwidth, showstringspaces=false}
\lstset{frame=trBL}

\begin{document}
\maketitle

\section{Lu Decomposition}

\subsection{Teoria}

\subsection{Jama}

\subsection{Programma lu-decomposition}
\lstinputlisting[caption=lu-decompostion,label=lst:java]{../../lu-decomposition/src/Main.java}

\subsection{Risultati e conclusioni}

\begin{table}
\begin{tabular}{|l|r|r|r|r|r|}
TEST         &	ERRORE RELATIVO  &	ERRORE PRIMA COMP  & AUTOVALORE N 7    &	TIME TO SOLVE &	TIME TO EIGEN \\
\hline
     easy-10 &	    3.510833e-16 &	      2.220446e-16 &           7.000000 & 	0ms		 	 &		0ms   \\
    easy-100 &	    2.853360e-15 &	      1.110223e-15 &           7.000000 & 	1ms		 	 &		0ms   \\
   easy-1000 &	    3.174443e-14 &	      3.264056e-14 &           7.000000 &	670ms		 &		9ms   \\
     rand-10 &	    4.711062e-15 &	      8.659740e-15 &               n.a. & 	0ms		 	 &		0ms   \\
    rand-100 &	    1.001901e-13 &	      8.237855e-14 &               n.a. & 	1ms		 	 &		0ms   \\
   rand-1000 &	    2.547025e-12 &	      4.767298e-13 &               n.a. &	647ms		 &		0ms   \\
   rand-5000 &	    9.787832e-12 &	      1.521339e-11 &               n.a. &	101418ms	 &		0ms   \\
      bad-10 &	    3.118816e-07 &	      2.176155e-07 &           6.000000 & 	0ms		 	 &		0ms   \\
     bad-100 &	    2.258293e-05 &	      2.394252e-05 &           6.000000 & 	1ms		 	 &		0ms   \\
     bad-500 &	    4.622912e-05 &	      4.654016e-05 &           6.000000 &	69ms		 &		2ms   \\
    bad-1000 &	    2.279306e-04 &	      2.257559e-04 &           6.000000 &	645ms		&		13ms   \\
  verybad-10 &	    4.993346e-04 &	      4.179128e-04 &           6.000000 & 	0ms		 	&		0ms   \\
 verybad-100 &	    3.283544e-03 &	      3.125627e-03 &           6.000000 & 	2ms		 	&		0ms   \\
 verybad-500 &	    1.065932e-02 &	      1.067230e-02 &           6.000000 &	70ms		 &		1ms   \\
verybad-1000 &	    2.926093e-02 &	      2.903832e-02 &           6.000000 &	644ms		 &		8ms   \\
      eig-10 &	            n.a. &	              n.a. &           1.212788 & 	0ms		 	&		0ms   \\
      eig-20 &	            n.a. &	              n.a. &           0.616452 & 	0ms		 &			0ms   \\
      eig-30 &	            n.a. &	              n.a. &           0.386165 & 	0ms		 &			0ms   \\
      eig-40 &	            n.a. &	              n.a. &           0.251158 & 	0ms		 &			0ms   \\
      eig-50 &	            n.a. &	              n.a. &           0.183589 & 	0ms		 &			0ms   \\
     eig-100 &	            n.a. &	              n.a. &           0.105820 & 	0ms		 &			1ms   \\
    eig-1000 &	            n.a. &	              n.a. &           0.005791 & 	0ms		 &			7ms   \\
    eig-2000 &	            n.a. &	              n.a. &           0.004094 & 	0ms		&			30ms   \\
    eig-5000 &	            n.a. &	              n.a. &           0.001635 &	0ms		&			967ms   \\
\end{tabular}
\end{table}

\section{Discrete Cosine Transform}

\subsection{Teoria}

\subsection{JTransform}

\subsection{Programma discrete-cosine-transform}
\lstinputlisting[caption=discrete-cosine-transform,label=lst:java]{../src/Dct.java}

\subsection{Risultati e conclusioni}

\end{document}
